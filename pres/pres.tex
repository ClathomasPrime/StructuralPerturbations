\documentclass[usenames,dvipsnames]{beamer}

%\usepackage{listings}
\usepackage{colortbl}% http://ctan.org/pkg/xcolor
\usepackage{tikz}
\usepackage{tikz-cd}
%\usepackage{natbib}
%\usepackage{bibentry}
\usepackage{color}
\usepackage{mathtools}
\usepackage{qtree}
\usepackage{rotating}
\usepackage[customcolors]{hf-tikz}
% \usepackage{tabularx}
\usetikzlibrary{positioning,chains,fit,shapes,calc}
\usetikzlibrary{shapes.geometric, arrows}
\usetikzlibrary{automata}

\usepackage[noend]{algpseudocode}
\usepackage{stmaryrd} %for \varoast (circle with asterisk)

\usetheme{Pittsburgh}

\newcommand{\e}{\emptyset}
\newcommand{\req}[0]{\begin{turn}{90} $=$\end{turn}}

\newcommand{\emphasize}[1]{{\color[rgb]{1,0,0}#1}}
\newcommand{\emphasizeWhen}[2]{ {\color<#1>[rgb]{1,0,0} #2} }
\newcommand{\F}{\mathbb{F}}

\newcommand{\X}{{\!\!\color{red}\textbf{X}\!\!}}
\newcolumntype{d}{>{\columncolor[rgb]{0,0.5,0.5}}c}
\newcolumntype{P}{>{\columncolor[rgb]{0,0.5,0}}c}

\definecolor{light-gray}{gray}{0.95}

\title{Verifying Robustness of Programs Under Structural Perturbations}
\author{Clay Thomas and Jacob Bond}
\date{\today}

% \titlegraphic{
%   \includegraphics[width=0.4\textwidth]{Images/Purdue}
% }

\begin{document}

  % \bibliographystyle{alpha}
  % \nobibliography{robustness}

\begin{frame}
  \titlepage
\end{frame}

\begin{frame}[fragile]{Motivation}
    \begin{itemize}
        \item<1-> An attempt to synthesize the max function using PBE:
            \begin{itemize}
                \item \((13, 15) \mapsto 15\)
                \item \((-23, 19) \mapsto 19\)
                \item \((-75, -13) \mapsto -13\)
            \end{itemize}
        \item<2-> Synthesized program: \verb!P(a,b):=return b!
        \item<3-> Neither synthesized program, nor synthesizer are {\it robust}
    \end{itemize}
\end{frame}

\begin{frame}[fragile]{Robustness}
    \begin{itemize}
        \item<1-> Robustness: behaving predictably on uncertain inputs \cite{chaudhuri12}
        \item<2-> \verb!P(13,15)! \(\not=\) \verb!P(15,13)!
        \item<3-> 
            \begin{itemize}
                \item \((15, 13) \mapsto 15\)
                \item \((19, -23) \mapsto 19\)
                \item \((-13, -75) \mapsto -13\)
            \end{itemize}
            would synthesize very different program
        \item<4-> Synthesize a robust program or develop robust synthesizer
    \end{itemize}
\end{frame}

\begin{frame}[fragile]{Robustness Properties}
    \begin{itemize}[<+->]
        \item {\bf Continuity}: small change to input \(\Rightarrow\) small change to output
        \begin{center}
            \verb!Sort([1,4,3,6])=[1,3,4,6]!
            \verb!Sort([2,3,3,5])=[2,3,3,5]!
        \end{center}
%    \item Permutation: permuting input permutes the output
%        \begin{center}
%            \verb!Find([1,4,3,6], 4)=1!
%
%            \(\sigma = (0\mapsto2, 1\mapsto3, 2\mapsto1, 3\mapsto0)\)%(0\,2\,1\,3)\)
%
%            \verb!Find([6,3,1,4], 4)=3!
            
%            \(\operatorname{Find}(\sigma[1,4,3,6]) = \sigma\operatorname{Find}([1,4,3,6])\)
%        \end{center}
    \item {\bf Permutation}: permuting input leaves output invariant
        \begin{center}
            \verb!Sort([1,4,3,6])=[1,3,4,6]!
            \verb!Sort([6,3,1,4])=[1,3,4,6]!
        \end{center}
    \item {\bf Simultaneous Permutation}: permuting all inputs leaves output invariant (\verb!Grade(responses,answers)!)
        \begin{center}

            \verb!Grade([sqrt(x^2), 1/e, 6.5], [abs(x), e^-1, 13/2])=1!

            rearrange problem parts

            \verb!Grade([1/e, 6.5, sqrt(x^2)], [e^-1, 13/2, abs(x)])=1!
        \end{center}
    \end{itemize}
\end{frame}

\begin{frame}[fragile]{Verifying Continuity \cite{chaudhuri10}}
    \begin{itemize}
        \item<1-> Consider
            \begin{algorithmic}[1]
                \If{\(x \geq 0\)}
                    \State \(r := y\)
                \Else
                    \State \(r := z\)
                \EndIf
            \end{algorithmic}
        \item<2-> If \(y \not= z\), discontinuous at \(x = 0\)
        \item<3-> Proof rule:
            \begin{align*}
                c \vdash \mathrm{Cont}(P_{1}, \mathrm{In}, \mathrm{Out}) &\qquad c \vdash \mathrm{Cont}(P_{2}, \mathrm{In}, \mathrm{Out})\\
                c' \vdash \mathrm{Cont}(b, \mathrm{Var}(b)) &\qquad (c \wedge \neg c') \vdash \mathrm{Out}_{P_{1}} = \mathrm{Out}_{P_{2}}
            \end{align*}
            \vspace{-2\baselineskip}
            \begin{center}
                \rule{0.85\textwidth}{0.5pt}
            \end{center}
            \vspace{-0.75\baselineskip}
            \begin{align*}
                c \vdash \mathrm{Cont}(\mathbf{if}\: b\: \mathbf{then}\: P_{1}\: \mathbf{else}\: P_{2}, \mathrm{In}, \mathrm{Out})
            \end{align*}
        \item<4-> Only applicable to numerical perturbations
    % focus on loop free programs
    % discuss if rules
    % only applies to numerical perturbations
    \end{itemize}
\end{frame}

%\begin{frame}[fragile]{If-Elsif-Elsif-Else Cost}
%    \begin{itemize}
%        \item<1->
%            \begin{align*}
%                c \vdash \mathrm{Cont}(P_{1}, \mathrm{In}, \mathrm{Out}) &\qquad c \vdash \mathrm{Cont}(P_{2}, \mathrm{In}, \mathrm{Out})\\
%                c' \vdash \mathrm{Cont}(b, \mathrm{Var}(b)) &\qquad (c \wedge \neg c') \vdash \mathrm{Out}_{P_{1}} = \mathrm{Out}_{P_{2}}
%            \end{align*}
%            \vspace{-2\baselineskip}
%            \begin{center}
%                \rule{0.85\textwidth}{0.5pt}
%            \end{center}
%            \vspace{-0.75\baselineskip}
%            \begin{align*}
%                c \vdash \mathrm{Cont}(\mathbf{if}\: b\: \mathbf{then}\: P_{1}\: \mathbf{else}\: P_{2}, \mathrm{In}, \mathrm{Out})
%            \end{align*}
%        \item<2-> Given \verb!if!-\verb!elsif!-\verb!elsif!-\verb!else!,
%            \begin{itemize}
%                \item<3-> Reason about each branch (4 branches)
%                \item<4-> Check equivalence (3 comparisons)
%            \end{itemize}
%    \end{itemize}
%\end{frame}

\begin{frame}{Cartesian Hoare Logic \cite{sousa16}}
    \begin{itemize}
        \item<1-> Robustness requires two executions
        \item<2-> Verified using product program
            \begin{itemize}
                \item<3-> \(P_{1} \varoast P_{2}\) is simultaneous execution
            \end{itemize}
        \item<4-> Cartesian Hoare Logic reasons about product programs
        \item<5-> Cartesian Hoare Triple examples:
            \begin{itemize}
                \item<6-> Determinism: \[\|\vec{x_{1}} = \vec{x_{2}}\|f(\vec{x})\|ret_{1} = ret_{2}\|\]
                \item<7-> Symmetry: \[\|x_{1} = y_{2} \wedge x_{2} = y_{1}\|f(x,y)\|ret_{1} = ret_{2}\|\]
            \end{itemize}
        \item<8-> Requires specifying property in first-order logic
        \item<9-> Not optimized for \(2\)-safety properties
    \end{itemize}
    % robustness as a 2-safety property
    % product program
    % Cartesian Hoare Logic
    % downside for many branches
    % specifications of properties
\end{frame}

%\begin{frame}[fragile]{If-Elsif-Elsif-Else Cost}
%    \begin{itemize}
%        \item<1-> Given \verb!if!-\verb!elsif!-\verb!elsif!-\verb!else!, must reason about product of each pair of branches (16 pairs):
%        \begin{gather*}
%        P_{1} \varoast P_{1}\\
%        \vdots\\
%        P_{1} \varoast P_{4}\\
%        P_{2} \varoast P_{1}\\
%        \vdots\\
%        P_{4} \varoast P_{4}
%        \end{gather*}
%    \end{itemize}
%\end{frame}

\begin{frame}[fragile]{Our Contributions}
  % \begin{itemize}
  %   % mention bug finding
  %   \item Small sets of perturbations that ``generate'' all perturbations
  %   \begin{itemize}
  %     \item Lists, binary search trees
  %   \end{itemize}
  %   \item Formulate ``invariance with respect to a function''
  %   \begin{itemize}
  %     \item General, sound procedure
  %   \end{itemize}
  %   \vfill
  %   \item Sanity checks and bug finding
  % \end{itemize}
  Goals:
  \begin{itemize}
      \item Reason about invariance under discrete perturbations
      \item Want to optimize for our specific problem
  \end{itemize}
      \vfill
  \uncover{Results:}
  \begin{itemize}
      \item Small sets of perturbations that ``generate'' all perturbations
      \begin{itemize}
        \item Lists, binary search trees
      \end{itemize}
      \item Formulate ``invariance with respect to a function''
      \begin{itemize}
        \item General, sound procedure
      \end{itemize}
      \item Sanity checks and bug finding
  \end{itemize}
\end{frame}

% Not a great transition
\begin{frame}[fragile]{Lists -- Invariance under order}
  Given an array $a$
  \begin{itemize}
    \item Let $a_{swap}$ be $a$ with its first and second entry swapped
    \begin{itemize}
      \item $[ a[1], a[0], a[2], a[3], \ldots, a[n] ]$
    \end{itemize}
    \item Let $a_{rot}$ be $a$ rotated by $1$
    \begin{itemize}
      \item $[ a[1], a[2], a[3], \ldots  a[n], a[0] ]$
    \end{itemize}
  \end{itemize}
  \vfill
  Lemma: If for any $a$, $P(a) = P(a_{swap}) = P(a_{rot})$,
  then for any permutation $a'$ of $a$,
  we have $P(a) = P(a')$.

  Proof: Math \cite{dummitfoote}
\end{frame}

% \begin{frame}[fragile]{Automata -- Invariance under order}
%   \begin{tikzpicture}[shorten >=1pt, node distance=1.5cm, on grid,
%       auto, scale=0.6, every node/.style={scale=0.6}
%   ] 
%     \node[state,initial] (q_0)   {$q_0$}; 
%     \node[state] (q_b) [right=of q_0] {$q_b$}; 
%     \node[state] (q_a) [above=of q_b] {$q_a$}; 
%     \node[state] (q_c) [below=of q_b] {$q_c$}; 
%     \node[state, accepting] (q_1) [right=of q_b] {$q_3$};
%     \path[->] 
%     (q_0) edge node {a / $\epsilon$} (q_a)
%           edge node {b / $\epsilon$} (q_b)
%           edge node [below left] {c / $\epsilon$} (q_c)
%     (q_a) edge node {x / xa} (q_1)
%     (q_b) edge node {x / xb} (q_1)
%     (q_c) edge node [below right] {x / xc} (q_1)
%     (q_1) edge [loop right] node {x / x} (q_1)
%           ;
%   \end{tikzpicture}
%   \ \ \ \ 
%   \begin{tikzpicture}[shorten >=1pt, node distance=1.5cm, on grid,
%       auto, scale=0.6, every node/.style={scale=0.6}
%   ] 
%     \node[state,initial] (q_0)   {$q_0$}; 
%     \node[state] (q_b) [right=of q_0] {$q_b$}; 
%     \node[state] (q_a) [above=of q_b] {$q_a$}; 
%     \node[state] (q_c) [below=of q_b] {$q_c$}; 
%     \node[state, accepting] (q_1) [right=of q_b] {$q_3$};
%     \path[->] 
%     (q_0) edge node {a / $\epsilon$} (q_a)
%           edge node {b / $\epsilon$} (q_b)
%           edge node [below left] {c / $\epsilon$} (q_c)
%     (q_a) edge node {\$ / a\$} (q_1)
%           edge [loop above] node {x / x} (q_a)
%     (q_b) edge node {\$ / b\$} (q_1)
%           edge [loop above] node {x / x} (q_b)
%     (q_c) edge node [below right] {\$ / c\$} (q_1)
%           edge [loop above] node {x / x} (q_c)
%           ;
%   \end{tikzpicture}
% \end{frame}
% 
% \begin{frame}[fragile]{Automata -- Invariance under order}
%   Theorem: Given a deterministic automata $M$, we can check if $M$
%   is invariant under the order of its input
%   in time $O(|\Sigma|^2|M|\log(|\Sigma||M|))$.
%   \vspace{0.3in}
% 
%   Proof:
%   \begin{itemize}
%     \item Construct the machines $M_{swap}$ and $M_{rot}$ by
%       composing $M$ with those machines
%     \begin{itemize}
%       \item Requires $O(|\Sigma||M|)$ states
%     \end{itemize}
%     \item Check if $L(M_{swap}) = L(M) = L(M_{rot})$
%     \begin{itemize}
%       \item State minimization in time $O(n|\Sigma| \log n)$
%         [Hopcroft 1971]
%     \end{itemize}
%   \end{itemize}
% \end{frame}

\begin{frame}[fragile]{Programs -- Invariance under order}
  \begin{itemize}
    \item maxList([x]) = x
    \item maxList([x, ...xs...]) = max(x, maxList(xs))
  \end{itemize}
  \vfill
  \begin{itemize}
    \item Verifying maxList($a$) = maxList($a_{swap}$) has one case:
  \end{itemize}
  \begin{align*}
    maxList([x, y, ...xs...])
      & \stackrel{?}{=}
    maxList([y, x, ...xs...])
    \\ || \qquad & \qquad || \\
    max(x, maxList([y, ...xs...]))
      & \qquad
    max(y, maxList([x, ...xs...]))
    \\ || \qquad & \qquad || \\
    max(x, max(y, maxList(xs)))
      & \qquad
    max(y, max(x, maxList(xs)))
    \\ || \qquad & \qquad || \\
    max(x, max(y, z))
      & \qquad
    max(y, max(x, z))
  \end{align*}
\end{frame}

\begin{frame}[fragile]{Binary Search Trees}
  \begin{itemize}
    \item For lists, two simple permutations generated all permutations
    \item Goal: similar permutations for BSTs
  \end{itemize}
\end{frame}

\begin{frame}[fragile]{Binary Search Trees}
  \begin{columns}
    \begin{column}{0.4\textwidth}
      \qroofx=2
      \qroofy=1
      \Tree [.b [.a \qroof{LL}. \qroof{LR}. ] \qroof{R}. ]
    \end{column}
    \begin{column}{0.1\textwidth}
      $\xmapsto{\mathrm{rotate}}$
    \end{column}
    \begin{column}{0.4\textwidth}
        \qroofx=2
        \qroofy=1
        \Tree [.a \qroof{LL}. [.b \qroof{LR}. \qroof{R}. ]  ]
    \end{column}
  \end{columns}
  \vfill
  \begin{columns}
    \begin{column}{0.45\textwidth}
      \qroofx=2
      \qroofy=1
      \Tree [.c [.a \qroof{LL}. [.b \qroof{LRL}. \qroof{LRR}. ]] \qroof{R}. ]
    \end{column}
    \begin{column}{0.1\textwidth}
      $\xmapsto{\mathrm{flatten}}$
    \end{column}
    \begin{column}{0.45\textwidth}
        \qroofx=2
        \qroofy=1
        \Tree [.c [.b [.a \qroof{LL}. \qroof{LRL}. ] \qroof{LRR}. ] \qroof{R}. ]
    \end{column}
  \end{columns}
  \vfill
\end{frame}

\begin{frame}[fragile]{Binary Search Trees}
  \begin{columns}
    \begin{column}{0.5\textwidth}
      It suffices to show
      \begin{itemize}
        \item Every tree can be transformed into a ``normal form'' (i.e. list)
        \begin{itemize}
          \item ``flatten'' straightens out the tree
          \item ``rotate'' lets you straighten all the parts
        \end{itemize}
        \item Every operation is reversable
      \end{itemize}
    \end{column}
    \begin{column}{0.5\textwidth}  %%<--- here
      \Tree [.1 $\e$ [.2 $\e$ [.3 $\e$ [.4 $\e$ [.5 $\e$ 6 ]]]]]
    \end{column}
  \end{columns}
\end{frame}

\begin{frame}[fragile]{Lists and Binary Search Trees}
  \begin{itemize}
    \item Can check robustness under ALL permutations
      by checking just TWO permutations
  \end{itemize}
\end{frame}

% \begin{frame}[fragile]{Binary Search Trees -- Proof by example}
%   \begin{itemize}
%     \item Every tree can be transformed into a ``normal form'' (i.e. list)
%   \end{itemize}
% 
%   \only<1-2>{
%     \Tree [.7 [.3 [.2 1 $\e$ ] [.5 4 6 ]] [.9 8 $\e$ ]]
%   }
%   \only<2>{
%     $\xmapsto{\mathrm{rotate}^{-1}}$
%   }
%   \only<2-3>{
%     \Tree [.9 [.7 [.3 [.2 1 $\e$ ] [.5 4 6 ]] 8 ] $\e$ ]
%   }
%   \only<3>{
%     $\xmapsto{\mathrm{flatten}}$
%   }
%   \only<3-4>{
%     \Tree [.9 [.8 [.7 [.3 [.2 1 $\e$ ] [.5 4 6 ]] $\e$ ] $\e$ ] $\e$ ]
%   }
%   \only<4>{
%     $\xmapsto{\mathrm{rotate}}$
%   }
%   \only<4-5>{
%     \Tree [.8 [.7 [.3 [.2 1 $\e$ ] [.5 4 6 ]] $\e$ ] 9 ]
%   }
%   \only<5>{
%     $\xmapsto{\mathrm{rotate}}$
%   }
%   \only<5-6>{
%     \Tree [.7 [.3 [.2 1 $\e$ ] [.5 4 6 ]] [.8 $\e$ 9 ] ]
%   }
%   \only<6>{
%     $\xmapsto{\mathrm{flatten}}$
%   }
%   \only<6-7>{
%     \Tree [.7 [.5 [.3 [.2 1 $\e$ ] 4 ] 6 ] [.8 $\e$ 9 ] ]
%   }
%   \only<7>{
%     $\xmapsto{\mathrm{flatten}}$
%   }
%   \only<7-8>{
%     \Tree [.7 [.6 [.5 [.3 [.2 1 $\e$ ] 4 ] $\e$ ] $\e$ ] [.8 $\e$ 9 ] ]
%   }
%   \only<8>{
%     $\xmapsto{\mathrm{rotate}}$
%   }
%   \only<8-9>{
%     \Tree [.6 [.5 [.3 [.2 1 $\e$ ] 4 ] $\e$ ] [.7 $\e$ [.8 $\e$ 9 ] ]]
%   }
%   \only<9>{
%     $\xmapsto{\mathrm{rotate}}$
%   }
%   \only<9-10>{
%     \Tree [.5 [.3 [.2 1 $\e$ ] 4 ] [.6 $\e$ [.7 $\e$ [.8 $\e$ 9 ] ]]]
%   }
%   \only<10>{
%     $\xmapsto{\mathrm{flatten}}$
%   }
%   \only<10-11>{
%     \Tree [.5 [.4 [.3 [.2 1 $\e$ ] $\e$ ] $\e$ ] [.6 $\e$ [.7 $\e$ [.8 $\e$ 9 ] ]]]
%   }
%   \only<11>{
%     $\xmapsto{\mathrm{rotate}}$
%   }
%   \only<11-12>{
%     \Tree [.4 [.3 [.2 1 $\e$ ] $\e$ ] [.5 $\e$ [.6 $\e$ [.7 $\e$ [.8 $\e$ 9 ] ]]]]
%   }
%   \only<12>{
%     $\xmapsto{\mathrm{rotate}}$
%   }
%   \only<12-13>{
%     \Tree [.3 [.2 1 $\e$ ] [.4 $\e$ [.5 $\e$ [.6 $\e$ [.7 $\e$ [.8 $\e$ 9 ] ]]]]]
%   }
%   \only<13>{
%      $\xmapsto{\mathrm{rotate}}$
%   }
%   \only<13-14>{
%     \Tree [.2 1 [.3 $\e$ [.4 $\e$ [.5 $\e$ [.6 $\e$ [.7 $\e$ [.8 $\e$ 9 ] ]]]]]]
%   }
%   \only<14>{
%     \hspace{-0.5in} $\xmapsto{\mathrm{rotate}}$
%   }
%   \only<14-15>{
%     \Tree [.1 $\e$ [.2 $\e$ [.3 $\e$ [.4 $\e$ [.5 $\e$ [.6 $\e$ [.7 $\e$ [.8 $\e$ 9 ] ]]]]]]]
%   }
% \end{frame}

% Maybe swap the code example for a list function
% Save this for the paper
% \begin{frame}[fragile]{Checking BST code}
%   \begin{itemize}
%     \item Goal: write a secondSmallest function on BSTs
%   \end{itemize}
%   secondSmallest$\Bigg($\hspace{-1in}\Tree [.b [.a $\e$ $\e$ ] \qroof{R}. ]$\Bigg)$ = b
%   \vfill
%   secondSmallest$\Bigg($\hspace{-0.7in}\Tree [.a $\e$ \qroof{R}. ]$\Bigg)$ = smallest(R)
%   \vfill
%   (more cases)
% \end{frame}
% 
% \begin{frame}[fragile]{Checking BST code}
%   \begin{itemize}
%     \item One case of checking invariance under rotation:
%   \end{itemize}
%   \begin{align*}
%     secondSmallest\Bigg(\Tree [.b [.a $\e$ $\e$ ] \qroof{R}. ] \Bigg) 
%       & = &
%     secondSmallest\Bigg(\Tree [.a  $\e$ [.b $\e$ \qroof{R}. ]] \Bigg) 
%      \\ 
%     b
%       & = &
%     smallest\Bigg(\Tree [.b $\e$ \qroof{R}. ] \Bigg) 
%   \end{align*}
% 
% \end{frame}

\begin{frame}[fragile]{More General Procedure}
  \begin{itemize}
    \item Sets of permutations are case-by-case
    \item Goal: formulation of invariance
    \begin{itemize}
      \item Useful
      \item Easy to code/express
      \item Checkable
    \end{itemize}
  \end{itemize}
\end{frame}

\begin{frame}[fragile]{More General Procedure}
  Invariance of a program $P : T\to Z$
  relative to a function $f : T \to T'$
  \begin{itemize}
    \item $f(t)$ gives a ``canonical representative'' of $t$
    \item For concreteness, $f = list : BST \to List$
  \end{itemize}
  \vfill
  Observation:
  The following are equivalent:
  \begin{itemize}
    \item $list(x) = list(y) \implies P(x) = P(y)$
    \item There exists a program $\widetilde{P} : Lists\to Z$
      such that $P(t) = \widetilde{P}(list(t))$
  \end{itemize}

  \[
    \begin{tikzcd}
      BSTs \arrow{rr}{P} \arrow{rd}{list} &   & Z \\
      & Lists \arrow[dashed]{ru}{\widetilde{P}} &   \\
    \end{tikzcd}
  \]
\end{frame}

\begin{frame}[fragile]{More General Procedure}
  \begin{itemize}
    \item Idea: Synthesize a witness to the invariance
    \begin{itemize}
      \item A function $\widetilde{P} : Lists \to Z$
    \end{itemize}
    \item $P$ and $list$ provide a \emph{full specification} of $\widetilde{P}$
    \item Counterexample guided inductive synthesis \cite{solar06}
  \end{itemize}

  \[
    \begin{tikzcd}
      BSTs \arrow{rr}{P} \arrow{rd}{list} &   & Z \\
      & Lists \arrow[dashed]{ru}{\widetilde{P}} &   \\
    \end{tikzcd}
  \]
\end{frame}

\begin{frame}[fragile]{More General Procedure}
  \tikzstyle{process} = [rectangle, minimum width=3cm, minimum height=1cm,
    text centered, draw=black, fill=orange!30,
    text width=10em ]
  \tikzstyle{smallprocess} = [rectangle, minimum width=3cm, minimum height=1cm,
    text centered, draw=black, fill=orange!30,
    text width=6em ]
  \tikzstyle{io} = [rectangle, rounded corners,
    minimum width=3cm, minimum height=1cm,
    text centered, draw=black, fill=blue!30]
  \tikzstyle{arrow} = [thick,->,>=stealth]

  \begin{tikzpicture}[node distance=2cm, scale=0.8, every node/.style={scale=0.8}]
    \node (start) [io] {
      Input $P$.
      Let $E = \{ \}$.
    };
    \node (synth) [process, below of=start, yshift=-0.5cm] {
      \textbf{Synthesize} a program $\widetilde{P} : Lists \to Z$
      given examples $\{ (list(t), P(t)) | t\in E \}$
    };
    \node (verify) [process, right of=synth, xshift=4cm] {
      \textbf{Verify} whether $\widetilde{P}(list(t))$
      is equivalent to $P(t)$
    };
    \node (yes) [io, above of=verify] {
      return YES
    };
    \node (check) [process, below of=verify, yshift=-1cm] {
      Check if there are any $t\in E$ such that
      $list(t)=list(t_0)$, yet $P(t) \ne P(t_0)$
    };
    \node (no) [io, below of=check, yshift=-0.5cm] {
      return NO
    };
    \node (add) [smallprocess, below of=synth, yshift=-1cm] {
      Add $t_0$ to the example set $E$
    };

    \draw [arrow] (start) -- (synth);
    \draw [arrow] (synth) -- (verify);
    \draw [arrow] (verify) --
      node[anchor=west, text width=6cm] {No, with \\ counterexample $t_0\in BSTs$}
      (check);
    \draw [arrow] (verify) --
      node[anchor=west, text width=6cm] {Yes, equivalent}
      (yes);
    \draw [arrow] (check) --
      node[anchor=north, text width=2cm, text centered] {No, $t$ does not exists}
      (add);
      \draw [arrow] (add) -- (synth);
    \draw [arrow] (check) --
      node[anchor=west, text width=6cm] {Yes, $t$ exists}
      (no);
  \end{tikzpicture}
\end{frame}

\begin{frame}{Future Directions}
    \begin{itemize}% [<+->]
        \item Develop proof rules for discrete perturbations
        \item Improved handling of branching programs by Cartesian Hoare Logic
        \item Working implementation of Cartesian Hoare Logic
        \item Find more data structures with small perturbation sets
        \item Speed up our general procedure
        \item Synthesis for verification?
        \item Implement!
    \end{itemize}
\end{frame}

\begin{frame}{References}
  \footnotesize
  \bibliographystyle{abbrv}
  \bibliography{robustness}
\end{frame}

\end{document}


% \begin{frame}[fragile]{Multicolumn Example}
%   \begin{columns}
%     \begin{column}{0.5\textwidth}
%       Hello
%     \end{column}
%     \begin{column}{0.5\textwidth}  %%<--- here
%       \begin{flushright}
%       image
%       \end{flushright}
%     \end{column}
%   \end{columns}
% \end{frame}

% \begin{frame}[fragile]{The Code Topology $T(1,1,1)$}
%   View codewords as a grid
%   \vfill
%   \begin{tabular}{c c c >{\columncolor[rgb]{.8,.8,1}}c | c}
%     $x_{1,1}$ & $x_{1,2}$ & $x_{1,3}$ & $x_{1,1} + x_{1,2} + x_{1,3}$ \\
%     $x_{2,1}$ & $x_{2,2}$ & \cellcolor[rgb]{1,0,0}$y$ & $x_{2,1} + x_{2,2} + y$ \\
%     \rowcolor[rgb]{.8,.8,1}
%     $x_{1,1}+x_{2,1}$ & $x_{1,2}+x_{1,2}$ & $x_{1,3}+y$ & (*) & (column parities)\\
%     \hline
%     & & & (row parities)
%   \end{tabular}
%   \vfill
%   Parities for correlated failures
%   \hfill
%   \emphasizeWhen{2}{
%     Maximize Recoverability
%   }

%   $(*) = \sum_{i,j} x_{i,j} $
%   \hfill
%   \emphasizeWhen{2}{
%     $y = \sum_{i,j} \gamma_{i,j} x_{i,j}$
%   }

%   \hfill
%   \emphasizeWhen{2}{
%     Global ``backup" parity
%   }

% \end{frame}
